\documentclass[10pt]{article}
%\usepackage{url}
%\usepackage{algorithmic}
\usepackage[margin=1in]{geometry}
\usepackage{datetime}
\usepackage[margin=2em, font=footnotesize]{caption}
\usepackage{graphicx}
\usepackage{mathpazo} % use palatino
\usepackage[scaled]{helvet} % helvetica
\usepackage{microtype}
\usepackage{amsmath}
\usepackage{subfigure}
\usepackage{listings}
\usepackage{wrapfig}
\usepackage{ amssymb }
% Letterspacing macros
\newcommand{\spacecaps}[1]{\textls[200]{\MakeUppercase{#1}}}
\newcommand{\spacesc}[1]{\textls[50]{\textsc{\MakeLowercase{#1}}}}
\lstdefinestyle{myCustomMatlabStyle}{
  language=Python,
  stepnumber=1,
  numbersep=10pt,
  tabsize=4,
  showspaces=false,
  showstringspaces=false
}
\lstset{basicstyle=\footnotesize,style=myCustomMatlabStyle}
\title{\flushleft2.4 Sensitivity of Optimisation Algorithms to Initialisation\\ }
\date{}
\usepackage{dsfont}
\usepackage{varwidth}
\usepackage{float}
\usepackage{bm}
\usepackage{mathtools}

\begin{document}
\hfill\framebox{\parbox[t][5 true cm][c]{11 true cm}
{\hfil Space for project label}}
\section*{\LARGE{9.3 Protein Comparison in Bioinformatics}}
\section*{1 The Edit Distance}
\underline{Definitions:}\\
\textit{S} and \textit{T} are strings of length \textit{m} and \textit{n} respectively.\\
We write $S_i$ for the \textit{i}th caracter of \textit{S}, and $S[i,j]$ for the substring $S_i,\dots,S_j$.\\
The \textit{optimal} edit transcripts are those that involve the least number of edit operations (\textbf{R}eplace, \textbf{I}nsert, \textbf{D}elete).\\
The \textit{edit distance} $d(S,T)$ is defined to be the minimal number of edits between \textit{S} and \textit{T}.\\
And we let $D(i,j)=d(S[1,i],T[1,j])$.
\subsection*{Question 1}
Assume that we have a string \textit{S} of length $i$ and a string \textit{T} of length $j$. We consider the different cases depending on the last letter of both strings.\\
Then for $i,j>0$,
\begin{itemize}
    \item \textbf{case 1: $S_i=T_j$}
    \begin{itemize}
    \item Then we ignore the last character, and no extra operations needs to be done,\\ so $D(i,j)=D(i-1,j-1)$
    \end{itemize}
    \item \textbf{case 2: $S_i \neq T_j$}
    \begin{itemize}
        \item Then we can try either of the three operations and see which one is more optimal in edit distance.
        \begin{enumerate}
            \item Replace: we can replace $S_i$ with $T_j$ and we again obtain the same situation as in case 1. Hence, we ignore last character and $D(i,j)=1+D(i-1,j-1)$.
            \item Insert: we can insert a character equal to $T_j$ to the end of $S$, so now $S$ has the length $i+1$. We now obtain the same situation as in case 1, hence we ignore last character and $D(i,j)= 1 + D(i,j-1)$.
            \item Delete: we can delete $S_i$ and we have $D(i,j)=1+D(i-1,j)$.
        \end{enumerate}
        \item we can then find the optimal edit distance by taking the minimum of these three choices.
    \end{itemize}
\end{itemize}
Therefore, we get
\[D(i,j)=min\{D(i-1,j)+1,\,D(i,j-1)+1,\,D(i-1,j-1)+s(S_i,T_j)\}\]
\[  s(S_i,T_j)=
    \begin{dcases}
        0 & S_i = T_j \\
        1 & S_i \neq T_j\\
    \end{dcases}
\]
\newpage
\noindent We are left to consider the boundary conditions when $i,j$ reaches $0$, and we'd be done with the recursion equation.
\begin{itemize}
    \item $D(0,0)=0$ -two empty strings are the same
    \item $D(i,0)=i$ -all characters from $S$ will be removed
    \item $D(0,j)=j$ -all characters from $T$ will be inserted into $S$
\end{itemize}
\bigskip
\subsection*{Question 2}
We use the following script to find the edit distance.
\begin{lstlisting}
import edit_distance as *
D = edit_distance_without_transcript('sheshells', 'seashells')
print(D[-1][-1])
\end{lstlisting}
The function used above can be found in the programs section under the title \emph{'i) edit\textunderscore distance.py'}.\\\\
The edit distance is \underline{$\bm{3}$} for \underline{shesells} and \underline{seashells}.\\
The spatial complexity of the algorithm above is $\mathcal{O}(mn)$ as it has a table $D$ to store the results of size $m*n$ and two extra constants $m$ and $n$.\\
In the algorithm, we used a for loop with $m$ iterations from $i=0,\dots,m$, and in each iteration, we used another for loop with $n$ iterations from $j=0,\dots,n$. At each individual iteration, we are doing:
\begin{itemize}
    \item 1 comparison and 1 assignment if $S[i]=T[j]$
    \item otherwise we are doing 3 comparisons, 3 additions and 1 assignment
\end{itemize}
These are all $\mathcal{O}(1)$ operations.\\
Hence, time complexity of the algorithm is $m\times n \times \mathcal{O}(1)=\mathcal{O}(mn)$
\bigskip
\subsection*{Question 3}
The script used to get the results is 
\begin{lstlisting}
from edit_distance import *
from proteins import *

results = edit_distance_without_transcript(A, B)
print(results[-1][-1])    #edit_distance
trace = backtrace(A, B)
print(trace[:50])   #first 50 steps of an optimal alignment
\end{lstlisting}
The functions used within can be found in the programs section under the title \emph{'i) edit\textunderscore distance.py'};\\ and the proteins data given on CATAM website can be found in the files section under the title \emph{'ii) proteins.py'}.\\\\
The edit distance is \underline{$\bm{83}$} and the first 50 steps of an optimal alignment is\\ \underline{\textit{MDDRMDDRRRMMMRMMRMMRMRRRRMRRMMRMMMMRMMMMRRRMRMMRMR}}.\\

\newpage
\section*{2 Scoring Matrix}
\subsection*{Question 4}
The script used to get the results below is
\begin{lstlisting}
from edit_distance_with_scoring_matrix import *
from proteins import *
from BLOSUM_matrix import *

results = edit_distance_without_transcript(A, B, -8, -8, blosum)
print(results[-1][-1][0])    #max score
trace = backtrace(A, B, -8, -8, blosum)
print(trace[:50])   #first 50 steps of an optimal alignment
\end{lstlisting}
The functions used above can be found in the programs section under \emph{'ii) edit\textunderscore distance\textunderscore with\textunderscore scoring\textunderscore matrix.py'}. (The name of the function is the same as in programs used in section 1 but with extra arguments)\\
The data files used above can be found in the files section under \emph{'i) BLOSUM\textunderscore matrix.py'} and \emph{'ii) proteins.py'} respectively. The Blosum file is equivalent to the ersion on CATAM website but I wrote it as a dictionary in python.\\\\
The max score for proteins $A$ and $B$ is \underline{\textbf{290}} and the first 50 steps of the optimal transcript is\\
\underline{\textit{MDDRMDDRRRMMMRMMRMMRMRRRRMRRMMRMMMMRMMMMRRRMRMMRMR}}.\\

\section*{3 Scoring for Gaps}
\subsection*{Question 5}
We note the that this is a generalisation of the methods we've developed in previous questions where $E(i,j)$ is the insertion of sequence of amino acids of S, $F(i,j)$ is the deletion of sequence of amino acids of S and G is the replacement of $S_i$ with $T_j$. Hence, the boundary conditions are:
\begin{itemize}
    \item $V_{gap}(0,0)=0$
    \item $V_{gap}(0,j)=V_{gap}(i,0)=u$ where $i,j>0$. This is because $\forall l\geq1$, we have $w(l)=u$, which means the weighted cost to delete or insert a string of length $i/j$ is $u$ for all $i$ and $j$.
\end{itemize}
Furthermore, note that $w(l)=u$ for all values of $l\geq1$ implies
\begin{flalign*}
E(i,j)&=max_{0\leq k\leq j-1}\{V_{gap}(i,k)+u\}&\\
&=max\{max_{0\leq k\leq j-2}V_{gap}(i,k)+u, V_{gap}(i,j-1)+u\}\\
&= \underline{max\{E(i,j-1),V_{gap}(i,j-1)+u\}}\quad (1)\\ 
\intertext{Similarly, we can rewrite equation for $F(i,j)$ as}
F(i,j)&= \underline{max\{F(i-1,j),V_{gap}(i-1,j)+u\}} \quad (2)
\end{flalign*}
Therefore, we require further boundary conditions on $E(i,0)$ and $F(0,j)$, and we can set \underline{$E(i,0)=F(0,j)=2u$}. Whilst $E(i,0)$ and $F(0,j)$ are not well defined since we can't edit a sequence of length i to make it length 0 through insertion nor can we edit a sequence length 0 to make it length i through deletion. The purpose of these boundary conditions is to make sure such that $E(i,0)=V_{gap}(i,0)+u$ and $F(0,j)=V_{gap}(0,j)+u$, so the values of $E(i,1)$ and $F(1,j)$ are correct. (In fact setting $E(i,0)$ and $F(0,j)$ to any value $\leq 2u$ would work too)\\
The algorithm works by initialising tables $E,F,Vgap$ which stores results of $E(i,j),F(i,j)$ and $V_{gap}(i,j)$ respectively. These tables would all have size $(m+1)\cdot (n+1)$. Then using the boundary conditions described above, we set up two for loops iterating through array of $(i,j)$,i.e. a for loop for $i=1,\dots,m$ and in each iteration of the $i$ loop, there's a for loop for $j=1,\dots,n$, so there are $mn$ individual iterations. In each such iteration, we calculate $E(i,j)$ and $F(i,j)$ using formula (1) and (2) listed above, where we need to perform 1 addition and 1 comparison. We then calculate $G(i,j)$ explicitly, which requires 1 addition; and we finally compare $E(i,j),F(i,j)$ and $G(i,j)$ to find $V_{gap}(i,j)$, which requires 2 comparisons. The result we need is just $Vgap(m,n)$.\\
Hence, time complexity at each iteration is $\mathcal{O}(1)$. Therefore, time complexity for the algorithm is $mn \times \mathcal{O}(1)=\mathcal{O}(mn)$.\\
The implementation of this algorithm is can be found in the programs section under \emph{'iii) edit\textunderscore distance\textunderscore with\textunderscore gaps.py'}.

\subsection*{Question 6}
The script used to get the results is
\begin{lstlisting}
from edit_distance_with_gaps import *
from proteins import *
from BLOSUM_matrix import *

results = edit_distance_without_transcript(C, D, -12, blosum)
print(results[-1][-1][0])
print(backtrace(C,D,-12,blosum)[:50])
\end{lstlisting}
The functions used above can be found in the programs section under \emph{'iii) edit\textunderscore distance\textunderscore with\textunderscore gaps.py'}. (The name of the functions is the same as in programs used in section 1,2 but from different files)\\
The data files used above can be found in the files section under \emph{'i) BLOSUM\textunderscore matrix.py'} and \emph{'ii) proteins.py'} respectively.\\\\
The score for proteins C and D is \underline{1236} and the first 50 steps of the optimal transcript is \\ \underline{MMDDDDDDDDDDDDDDDDDDDDDMRRRMRRRRRMRRDRMRRRRMRRMMRD}.


\section*{4 Statistical Significance}
\subsection*{Question 7}
$U^n$ and $V^n$ are independent and identically distributed. The score of inserting/deleting a sequence of length l is fixed: $w(l)=u$ for all $l\geq 1$.\\
We first consider what $\mathds{E}(v_{gap}(U^1,V^1))$ is. We observe that there are four possible cases:
\begin{flalign*}
&\left
\{\begin{array}{lr}
U=a,V=a;\\
U=a,V=b;\\
U=b,V=a;\\
U=b,V=b
\end{array}
\right.&\\
\intertext{Since $u\leq0$, $v_{gap}(U^0,V^0)=0$ and $v_{gap}(U^1,V^0)=v_{gap}(U^0,V^1)=u$, we get}
&\left
\{\begin{array}{lr}
v_{gap}(a,a)=max(2u,2u,1)=1\\
0\geq v_{gap}(a,b)=max(2u,2u,-1)\geq-1;\\
0\geq v_{gap}(b,a)=max(2u,2u,-1)\geq-1;\\
v_{gap}(b,b)=1
\end{array}
\right.
\end{flalign*}
Hence,
\begin{flalign*}
\mathds{E}(v_{gap}(U^1,V^1))&=p^2+(1-p)^2 + 2p(1-p)\cdot max(2u,-1)&\\
&\geq(p-(1-p))^2\\
&=(2p-1)^2\geq0
\end{flalign*}
We then observe
\begin{flalign*}
\mathds{E}(v_{gap}(U^n,V^n))&=\mathds{E}(max\{E(U^n,V^n),F(U^n,V^n),G(U^n,V^n)\})&\\
&\geq \mathds{E}(G(U^n,V^n))\\
&= \mathds{E}(v_{gap}(U^{n-1},V^{n-1})+s(U^n_n,V^n_n))\\
&= \mathds{E}(v_{gap}(U^{n-1},V^{n-1})) + (2p-1)^2
\end{flalign*}
Therefore, $\mathds{E}(v_{gap}(U^n,V^n))\geq n(2p-1)^2$. This implies that \[\frac{\mathds{E}(v_{gap}(U^n,V^n))}{n}\geq (2p-1)^2\]
and therefore \[\liminf_{n\to \infty}\frac{\mathds{E}(v_{gap}(U^n,V^n))}{n}\geq (2p-1)^2\geq 0 \qquad \text{for $0\leq p\leq1$}\]
What we've shown above is that if we use replacement only, the the $\mathds{E}(v_{gap}(U^n,V^n))=n(2p-1)^2$. And this would be a sufficient lower bound for all $p\neq \frac{1}{2}$.\\
However, we note that this lower bound can be further optimised. We do this by noting that for string segments $(U^{1-2u},V^{1-2u})$, the $v_{gap}$ score is always $\geq 2u$, since we can delete $U^{1-2u}$ and insert $V^{1-2u}$, which has a score of $2u$. We note that this method of operations is better than pure replacement if all $1-2u$ letters are not matched by $1$. (since $(1-2u)$ non-matched letter has a replacement score of $(2u-1)$)\\
So we can divide the strings into $\lfloor \frac{n}{1-2u}\rfloor$ disjoint sections, each of length $(1-2u)$, and check if they are made of only non-matching letters. The probability of this happening is $(2p(1-p))^{1-2u}$, and if they do we can increase the $v_{gap}$ score by $1$.\\
Therefore, we can update our lower bound as 
\begin{flalign*}
\mathds{E}(v_{gap}(U^n,V^n)) &\geq n(2p-1)^2 + 1 \times \lfloor \frac{n}{1-2u}\rfloor \times \mathds{E}(\mathds{1}_{\text{$1-2u$ consecutive non matched letters}})&\\
&\geq n(2p-1)^2 + \lfloor \frac{n}{1-2u}\rfloor \times (2p(1-p))^{1-2u}\\
&\geq n(2p-1)^2 + (\frac{n}{1-2u}-1) \times (2p(1-p))^{1-2u}\\
\frac{\mathds{E}(v_{gap}(U^n,V^n))}{n} &\geq (2p-1)^2 + \frac{(2p(1-p))^{1-2u}}{1-2u} - \frac{1}{n}(2p(1-p))^{1-2u}\\
\liminf_{n\to \infty}\frac{\mathds{E}(v_{gap}(U^n,V^n))}{n} 
&\geq  (2p-1)^2 + \frac{(2p(1-p))^{1-2u}}{1-2u} - \liminf_{n\to \infty}(\frac{1}{n}(2p(1-p))^{1-2u})
\end{flalign*}
Therefore, we may conclude that 
\[\liminf_{n\to \infty}\frac{\mathds{E}(v_{gap}(U^n,V^n))}{n} 
\geq  (2p-1)^2 + \frac{(2p(1-p))^{1-2u}}{1-2u}\]
because $\liminf_{n\to \infty}(\frac{1}{n}(2p(1-p))^{1-2u})=0$.\\
We note further $(2p-1)^2 + \frac{(2p(1-p))^{1-2u}}{1-2u} > 0$ for $p=\frac{1}{2}$, therefore $\forall p \in [0,1]$, 
\[\liminf_{n\to \infty}\frac{\mathds{E}(v_{gap}(U^n,V^n))}{n} > 0\]

\subsection*{Question 8}
The script used to find the limit can be found in the programs section under the title \emph{'iv) limit\textunderscore score.py'}. \\
By law of large number, we know that $\frac{1}{m}\sum_{i=1}^m v_{gap}(U^n_i,V^n_i)=\mathds{E}(v_{gap}(U^n,V^n))$, where $(U^n_i,V^n_i)$ are iid copies of $(U^n,V^n)$. So we can use Monte Carlo method, i.e. generate a large, random sample of $(U^n,V^n)$ and find their average to estimate $\mathds{E}(v_{gap}(U^n,V^n))$.\\
In my code, I used $m=500$ since this is a good balance between accuracy of $\mathds{E}(v_{gap}(U^n,V^n))$ and calculation time. And I tested this on a range of values of n, which are $(500, 1000, 1500, 2000, 2500)$. This is only to see the limit is roughly converging to. The results were $(0.420305,0.429393,0.432768,0.434200,0.435565)$. So the sequence is increasing.\\
I then tested with $n=5000$ and $m=500$, and the result was 0.438335. Calculating the difference between the expected score of different $n$, we note a significant decrease in the rate of change of the score.\\
When $n$ increases from $500$ to $1000$, the score increases by roughly $0.0090$, when $n$ increases from $1000$ to $2000$, the score increases by roughly $0.0050$. However, as $n$ increases from $2500$ to $5000$, the score only increases by roughly $0.00277$. So a very rough pattern is that when $n$ is doubled, the score increases by roughly $\frac{5}{9}$ of the previous increment.\\
$\sum_{n=1}^{\infty}(\frac{5}{9})^n=\frac{9}{4}-1=1.25$, hence, the limit of $n^{-1}\mathds{E}(V_{gap}(U^n,V^n))$ is around $0.438335+1.25*0.00277=0.4417975$. This should be accurate to 2d.p., so we get
\[\liminf_{n\to\infty} n^{-1}\mathds{E}(V_{gap}(U^n,V^n))\approx0.44\]

\section*{5 Local Alignment}
\subsection*{Question 9}
RHS = max\{$v_{sfx}(S',T'):S'$ a prefix of $S$, $T'$ a prefix of $
T\}$\\
$=max\{max\{v(S'',T''):$ $S''$ a suffix of $S'$, $T''$ a suffix of $T'$\} :$S'$ a prefix of $S$, $T'$ a prefix of $T$\}\\\\
To show that RHS = LHS = $v_{sub}(S,T)$, we need to show any pair of substrings of $S,T$ can be written as some $(S'',T'')$ such that $S''$ a suffix of $S'$, $T''$ a suffix of $T'$, $S'$ a prefix of $S$, $T'$ a prefix of $T$. We also need to show that all such $(S'',T'')$ are substrings of $S,T$.\\\\
Let $S,T$ be strings of length $m,n$ respectively. Consider an arbitrary pair of substrings, $S(i,j)$ and $T(k,l)$, we note that we can write $S(i,j)$ as a suffix of $S(1,j)$ which is a prefix of $S$. Similarly, we can write $T(k,l)$ as a suffix of $T(1,l)$ which is a prefix of $T$. Therefore, all pairs of substrings of $S$, $T$ can be written as $(S'',T'')$.\\
This implies $v_{sub}(S,T) \leq$ max\{$v_{sfx}(S',T'):S'$ a prefix of $S$, $T'$ a prefix of $T\}$, since the set of pairs of substrings of $S,T$ is a subset of the set of $(S'',T'')$.\\\\
On the other hand, let $S'$ be an arbitrary prefix $S(1,j)$ and $T'$ an arbitrary prefix $T(1,l)$. Then $S''$ is a suffix of $S'$, =$S(i,j)$ for some $i=1,\dots,j$ and $T''$ is a suffix of $T'$, =$T(k,l)$ for some $k=1,\dots,l$. Therefore, the set of $(S'',T'')$ is a subset of the set of pairs of substrings of $S,T$.\\
This implies $v_{sub}(S,T) \geq$ max\{$v_{sfx}(S',T'):S'$ a prefix of $S$, $T'$ a prefix of $T\}$\\\\
Combining these two, we get \underline{$v_{sub}(S,T)$ = max\{$v_{sfx}(S',T'):S'$ a prefix of $S$, $T'$ a prefix of $T\}$}

\subsection*{Question 10}
$V_{sfx}(i,j)=v_{sfx}(S[1,i],T[1,j])=max_{k,l}(S[k,i],T[l,j])$, where $1\leq k \leq i$ and $1 \leq l \leq j$. And we let the pair of the longest suffixes (i.e. where $i-k+j-l$ is maximised) that gives the maximum be $(S[k',i],T[l',j])$.\\ If one or both of these suffixes is an empty suffix, then $V_{sfx}(i,j)=0$. Else, both are non empty suffixes. And we know that there are three possible operations to edit a substring, namely insertion, deletion or replacement. So we get:
\begin{itemize}
    \item Insertion\\
    $v(S[k',i],T[l',j])=v(S[k',i],T[l',j-1])+s(-,T_j)$
    \item Deletion\\
    $v(S[k',i],T[l',j])=v(S[k',i-1],T[l',j])+s(S_i,-)$
    \item Replacement\\
    $v(S[k',i],T[l',j])=v(S[k',i-1],T[l',j-1])+s(S_i,T_j)$
\end{itemize}
We now make the importance observation that if $(S[k',i],T[l',j])$ is the longest maximum pair of suffixes for $(S[1,i],T[1,j])$, then $(S[k',i],T[l',j-1])$ must also be the longest maximum pair of suffixes for $(S[1,i],T[1,j-1])$.\\
To show this, assume it's not, then the pair must include at least one of $S[k'-1]$ or $T[l'-1]$, but then $(S[k',i],T[l',j])$ would no longer be the maximum pair of suffixes that's the longest for $(S[1,i],T[1,j])$ which is a contradiction.\\
The same deduction can be used for $(S[1,i-1],T[1,j])$ and $(S[1,i-1],T[1,j-1])$ as well.\\
Therefore, we can conclude that if the following operations took place then:
\begin{itemize}
    \item Insertion\\
    $V_{sfx}(i,j)=V_{sfx}(i,j-1)+s(-,T_j)$
    \item Deletion\\
    $V_{sfx}(i,j)=V_{sfx}(i-1,j)+s(S_i,-)$
    \item Replacement\\
    $V_{sfx}(i,j)=V_{sfx}(i-1,j-1)+s(S_i,T_j)$
\end{itemize}
We choose the operation that can maximise $V_{sfx}(i,j)$, but if none of the three operations provides a positive score, we may just take 2 empty suffixes which has a score of 0. Therefore, \[V_{sfx}(i,j)=max\left
\{\begin{array}{lr}
0,\\
V_{sfx}(i-1,j-1)+s(S_i,T_j),\\
V_{sfx}(i-1,j)+s(S_i,-),\\
V_{sfx}(i,j-1)+s(-,T_j)
\end{array}
\right.\]
For boundary conditions, we note that since $S(-,a)=S(a,-)<0$, for $(S[1,i],0)$ and $(0,T[1,j])$, the suffixes that maximises the score is the empty suffixes. Hence $V_{sfx}(i,0)=V_{sfx}(0,j)=0$


\subsection*{Question 11}
To find $v_{sub}$, it is sufficient to find the maximum of $V_{sfx}(i,j)$ for $1\leq i \leq m, 1\leq j\leq n$ as defined in question 10, since $V_{sfx}(i,j)$ is the $v_{sfx}$ score for the pair of prefixes $(S[1,i],T[1,j])$. And by question 9, $v_{sub}(S,T)$ = max\{$v_{sfx}(S',T'):S'$ a prefix of $S$, $T'$ a prefix of $T\}$.\\
The script to get the results is
\begin{lstlisting}
from edit_distance_with_local_alignment import *
from proteins import *
from BLOSUM_matrix import *

result = v_sub(C, D, -2, -2, blosum)
print(result)
\end{lstlisting}
The function used above is called 'v\textunderscore sub' and it can be found in the programs section under \emph{'v) edit\textunderscore distance\textunderscore with\textunderscore local \textunderscore alignment.py'}.\\\\
The $v_{sub}$ score for proteins C and D is \underline{1312}.


\newpage
\section*{Files}
\subsection*{i) BLOSUM\textunderscore matrix.py}
\lstdefinestyle{myCustomMatlabStyle}{
  language=Python,
  stepnumber=1,
  numbersep=10pt,
  tabsize=4,
  showspaces=false,
  showstringspaces=false
}
\lstset{basicstyle=\footnotesize,style=myCustomMatlabStyle}
\begin{lstlisting}
blosum = {('W', 'F'): 1, ('L', 'R'): -2, ('S', 'P'): -1, ('V', 'T'): 0, ('Q', 'Q'): 5, 
            ('N', 'A'): -2, ('Z', 'Y'): -2, ('W', 'R'): -3, ('Q', 'A'): -1, ('S', 'D'): 0, 
            ('H', 'H'): 8, ('S', 'H'): -1, ('H', 'D'): -1, ('L', 'N'): -3,('W', 'A'): -3,
            ('Y', 'M'): -1, ('G', 'R'): -2, ('Y', 'I'): -1, ('Y', 'E'): -2, ('B', 'Y'): -3,
            ('Y', 'A'): -2, ('V', 'D'): -3, ('B', 'S'): 0, ('Y', 'Y'): 7, ('G', 'N'): 0,
            ('E', 'C'): -4, ('Y', 'Q'): -1, ('Z', 'Z'): 4, ('V', 'A'): 0, ('C', 'C'): 9,
            ('M', 'R'): -1, ('V', 'E'): -2, ('T', 'N'): 0, ('P', 'P'): 7, ('V', 'I'): 3, 
            ('V', 'S'): -2, ('Z', 'P'): -1, ('V', 'M'): 1, ('T', 'F'): -2, ('V', 'Q'): -2, 
            ('K', 'K'): 5, ('P', 'D'): -1, ('I', 'H'): -3, ('I', 'D'): -3, ('T', 'R'): -1, 
            ('P', 'L'): -3, ('K', 'G'): -2, ('M', 'N'): -2, ('P', 'H'): -2,('F', 'Q'): -3, 
            ('Z', 'G'): -2, ('X', 'L'): -1, ('T', 'M'): -1, ('Z', 'C'): -3, ('X', 'H'): -1, 
            ('D', 'R'): -2, ('B', 'W'): -4, ('X', 'D'): -1, ('Z', 'K'): 1, ('F', 'A'): -2,
            ('Z', 'W'): -3, ('F', 'E'): -3, ('D', 'N'): 1, ('B', 'K'): 0, ('X', 'X'): -1, 
            ('F', 'I'): 0, ('B', 'G'): -1, ('X', 'T'): 0, ('F', 'M'): 0, ('B', 'C'): -3, 
            ('Z', 'I'): -3, ('Z', 'V'): -2, ('S', 'S'): 4, ('L', 'Q'): -2, ('W', 'E'): -3, 
            ('Q', 'R'): 1, ('N', 'N'): 6, ('W', 'M'): -1, ('Q', 'C'): -3, ('W', 'I'): -3, 
            ('S', 'C'): -1, ('L', 'A'): -1, ('S', 'G'): 0, ('L', 'E'): -3, ('W', 'Q'): -2, 
            ('H', 'G'): -2, ('S', 'K'): 0, ('Q', 'N'): 0, ('N', 'R'): 0, ('H', 'C'): -3,
            ('Y', 'N'): -2, ('G', 'Q'): -2, ('Y', 'F'): 3, ('C', 'A'): 0, ('V', 'L'): 1, 
            ('G', 'E'): -2, ('G', 'A'): 0, ('K', 'R'): 2, ('E', 'D'): 2, ('Y', 'R'): -2, 
            ('M', 'Q'): 0, ('T', 'I'): -1, ('C', 'D'): -3, ('V', 'F'): -1, ('T', 'A'): 0, 
            ('T', 'P'): -1, ('B', 'P'): -2, ('T', 'E'): -1, ('V', 'N'): -3, ('P', 'G'): -2, 
            ('M', 'A'): -1, ('K', 'H'): -1,('V', 'R'): -3, ('P', 'C'): -3, ('M', 'E'): -2, 
            ('K', 'L'): -2, ('V', 'V'): 4, ('M', 'I'): 1, ('T', 'Q'): -1, ('I', 'G'): -4,
            ('P', 'K'): -1, ('M', 'M'): 5, ('K', 'D'): -1, ('I', 'C'): -1, ('Z', 'D'): 1, 
            ('F', 'R'): -3,('X', 'K'): -1, ('Q', 'D'): 0, ('X', 'G'): -1, ('Z', 'L'): -3, 
            ('X', 'C'): -2, ('Z', 'H'): 0,('B', 'L'): -4, ('B', 'H'): 0, ('F', 'F'): 6, 
            ('X', 'W'): -2, ('B', 'D'): 4, ('D', 'A'): -2,('S', 'L'): -2, ('X', 'S'): 0, 
            ('F', 'N'): -3, ('S', 'R'): -1, ('W', 'D'): -4, ('V', 'Y'): -1,('W', 'L'): -2, 
            ('H', 'R'): 0, ('W', 'H'): -2, ('H', 'N'): 1, ('W', 'T'): -2, ('T', 'T'): 5,
            ('S', 'F'): -2, ('W', 'P'): -4, ('L', 'D'): -4, ('B', 'I'): -3, ('L', 'H'): -3, 
            ('S', 'N'): 1,('B', 'T'): -1, ('L', 'L'): 4, ('Y', 'K'): -2, ('E', 'Q'): 2,
            ('Y', 'G'): -3, ('Z', 'S'): 0, ('Y', 'C'): -2, ('G', 'D'): -1, ('B', 'V'): -3, 
            ('E', 'A'): -1, ('Y', 'W'): 2, ('E', 'E'): 5,('Y', 'S'): -2, ('C', 'N'): -3, 
            ('V', 'C'): -1, ('T', 'H'): -2, ('P', 'R'): -2, ('V', 'G'): -3,('T', 'L'): -1, 
            ('V', 'K'): -2, ('K', 'Q'): 1, ('R', 'A'): -1, ('I', 'R'): -3, ('T', 'D'): -1,
            ('P', 'F'): -4, ('I', 'N'): -3, ('K', 'I'): -3, ('M', 'D'): -3, ('V', 'W'): -3, 
            ('W', 'W'): 11, ('M', 'H'): -2, ('P', 'N'): -2, ('K', 'A'): -1, ('M', 'L'): 2, 
            ('K', 'E'): 1, ('Z', 'E'): 4, ('X', 'N'): -1, ('Z', 'A'): -1, ('Z', 'M'): -1,
            ('X', 'F'): -1, ('K', 'C'): -3, ('B', 'Q'): 0, ('X', 'B'): -1, ('B', 'M'): -3,
            ('F', 'C'): -2, ('Z', 'Q'): 3, ('X', 'Z'): -1, ('F', 'G'): -3,('B', 'E'): 1,
            ('X', 'V'): -1, ('F', 'K'): -3, ('B', 'A'): -2, ('X', 'R'): -1, ('D', 'D'): 6,
            ('W', 'G'): -2, ('Z', 'F'): -3, ('S', 'Q'): 0, ('W', 'C'): -2, ('W', 'K'): -3,
            ('H', 'Q'): 0, ('L', 'C'): -1, ('W', 'N'): -4, ('S', 'A'): 1, ('L', 'G'): -4,
            ('W', 'S'): -3, ('S', 'E'): 0, ('H', 'E'): 0, ('S', 'I'): -2, ('H', 'A'): -2,
            ('S', 'M'): -1, ('Y', 'L'): -1, ('Y', 'H'): 2,('Y', 'D'): -3, ('E', 'R'): 0,
            ('X', 'P'): -2, ('G', 'G'): 6, ('G', 'C'): -3, ('E', 'N'): 0,('Y', 'T'): -2,
            ('Y', 'P'): -3, ('T', 'K'): -1, ('A', 'A'): 4, ('P', 'Q'): -1, ('T', 'C'): -1,
            ('V', 'H'): -3, ('T', 'G'): -2, ('I', 'Q'): -3, ('Z', 'T'): -1, ('C', 'R'): -3, 
            ('V', 'P'): -2,('P', 'E'): -1, ('M', 'C'): -1, ('K', 'N'): 0, ('I', 'I'): 4, 
            ('P', 'A'): -1, ('M', 'G'): -3,('T', 'S'): 1, ('I', 'E'): -3, ('P', 'M'): -2, 
            ('M', 'K'): -1, ('I', 'A'): -1, ('P', 'I'): -3,('R', 'R'): 5, ('X', 'M'): -1,
            ('L', 'I'): 2, ('X', 'I'): -1, ('Z', 'B'): 1, ('X', 'E'): -1,('Z', 'N'): 0,
            ('X', 'A'): 0, ('B', 'R'): -1, ('B', 'N'): 3, ('F', 'D'): -3, ('X', 'Y'): -1,
            ('Z', 'R'): 0, ('F', 'H'): -1, ('B', 'F'): -3, ('F', 'L'): 0, ('X', 'Q'): -1,
            ('B', 'B'): 4}
# the matrix above only contains the lower triangle of the table,
# need to invert to also get the upper triangle
blosum_inversed = {}
for key in blosum.keys():
    index1, index2 = key[0], key[1]
    blosum_inversed[(index2, index1)] = blosum[key]
blosum.update(blosum_inversed)
\end{lstlisting}
\subsection*{ii) proteins.py}
\lstdefinestyle{myCustomMatlabStyle}{
  language=Python,
  stepnumber=1,
  numbersep=10pt,
  tabsize=4,
  showspaces=false,
  showstringspaces=false
}
\lstset{basicstyle=\footnotesize,style=myCustomMatlabStyle}
\begin{lstlisting}
A = """MGLSDGEWQLVLKVWGKVEGDLPGHGQEVLIRLFKTHPETLEKFDKFKGLKTEDEMKASA\
DLKKHGGTVLTALGNILKKKGQHEAELKPLAQSHATKHKISIKFLEYISEAIIHVLQSKH\
SADFGADAQAAMGKALELFRNDMAAKYKEFGFQG"""

B = """MADFDAVLKCWGPVEADYTTMGGLVLTRLFKEHPETQKLFPKFAGIAQADIAGNAAISAH\
GATVLKKLGELLKAKGSHAAILKPLANSHATKHKIPINNFKLISEVLVKVMHEKAGLDAG\
GQTALRNVMGIIIADLEANYKELGFSG"""

C = """MTSDCSSTHCSPESCGTASGCAPASSCSVETACLPGTCATSRCQTPSFLSRSRGLTGCLL\
PCYFTGSCNSPCLVGNCAWCEDGVFTSNEKETMQFLNDRLASYLEKVRSLEETNAELESR\
IQEQCEQDIPMVCPDYQRYFNTIEDLQQKILCTKAENSRLAVQLDNCKLATDDFKSKYES\
ELSLRQLLEADISSLHGILEELTLCKSDLEAHVESLKEDLLCLKKNHEEEVNLLREQLGD\
RLSVELDTAPTLDLNRVLDEMRCQCETVLANNRREAEEWLAVQTEELNQQQLSSAEQLQG\
CQMEILELKRTASALEIELQAQQSLTESLECTVAETEAQYSSQLAQIQCLIDNLENQLAE\
IRCDLERQNQEYQVLLDVKARLEGEINTYWGLLDSEDSRLSCSPCSTTCTSSNTCEPCSA\
YVICTVENCCL"""

D = """MPYNFCLPSLSCRTSCSSRPCVPPSCHSCTLPGACNIPANVSNCNWFCEGSFNGSEKETM\
QFLNDRLASYLEKVRQLERDNAELENLIRERSQQQEPLLCPSYQSYFKTIEELQQKILCT\
KSENARLVVQIDNAKLAADDFRTKYQTELSLRQLVESDINGLRRILDELTLCKSDLEAQV\
ESLKEELLCLKSNHEQEVNTLRCQLGDRLNVEVDAAPTVDLNRVLNETRSQYEALVETNR\
REVEQWFTTQTEELNKQVVSSSEQLQSYQAEIIELRRTVNALEIELQAQHNLRDSLENTL\
TESEARYSSQLSQVQSLITNVESQLAEIRSDLERQNQEYQVLLDVRARLECEINTYRSLL\
ESEDCNLPSNPCATTNACSKPIGPCLSNPCTSCVPPAPCTPCAPRPRCGPCNSFVR"""
\end{lstlisting}
\bigskip\bigskip\bigskip
\newpage
\section*{Programs}
\bigskip
\subsection*{i) edit\textunderscore distance.py}
\lstdefinestyle{myCustomMatlabStyle}{
  language=Python,
  stepnumber=1,
  numbersep=10pt,
  tabsize=4,
  showspaces=false,
  showstringspaces=false
}
\lstset{basicstyle=\footnotesize,style=myCustomMatlabStyle}
\begin{lstlisting}
# S, T are the two strings to be compared
def edit_distance_without_transcript(S, T):
    m = len(S)  # length of S
    n = len(T)  #length of T
    # initialising the results table
    D = [[0 for i in range(n+1)] for i in range(m+1)]
    # boundary conditions
    for i in range(m+1):
        D[i][0] = i
    for j in range(n+1):
        D[0][j] = j
    # recursion calculation
    for i in range(0, m):
        for j in range(0, n):
            if S[i] == T[j]:
                D[i+1][j+1] = D[i][j]
            else:
                D[i+1][j+1] = min(D[i][j+1]+1, D[i+1][j]+1, D[i][j]+1)
    return D

def backtrace(S, T):
    D = edit_distance_without_transcript(S, T)
    # backtracing for optimal route
    m = len(S)
    n = len(T)
    i, j = m,n
    string = ''
    while i > 0 or j > 0:
        # replace
        if i and j and D[i][j] == 1 + D[i-1][j-1]:
            i -= 1
            j -= 1
            string += 'R'
        # deletion
        elif i and D[i][j] == 1 + D[i-1][j]:
            i -= 1
            string += 'D'
        # insertion
        elif j and D[i][j] == 1 + D[i][j-1]:
            j -= 1
            string += 'I'
        # matching
        else:
            i -= 1
            j -= 1
            string += 'M'

    return string[::-1]
    
\end{lstlisting}
\newpage
\subsection*{ii) edit\textunderscore distance\textunderscore with\textunderscore scoring\textunderscore matrix.py}
\lstdefinestyle{myCustomMatlabStyle}{
  language=Python,
  stepnumber=1,
  numbersep=10pt,
  tabsize=4,
  showspaces=false,
  showstringspaces=false
}
\lstset{basicstyle=\footnotesize,style=myCustomMatlabStyle}
\begin{lstlisting}
# S, T are the two strings to be compared
def edit_distance_without_transcript(S, T, insertion, deletion, replacement):
    m = len(S)  # length of S
    n = len(T)  #length of T
    # initialising the results table
    D = [[[0, ''] for i in range(n+1)] for i in range(m+1)]
    # boundary conditions
    for i in range(m+1):
        D[i][0][0] = i * deletion
        D[i][0][1] = 'D'
    for j in range(n+1):
        D[0][j][0] = j * insertion
        D[0][j][1] = 'I'
    D[0][0][0] = 0
    D[0][0][1] = 'M'
    # recursion calculation
    for i in range(0, m):
        for j in range(0, n):
            D[i+1][j+1][0] = max(D[i][j+1][0]+deletion,
                D[i+1][j][0]+insertion, D[i][j][0]+replacement[(S[i], T[j])])
            if D[i+1][j+1][0] == D[i][j+1][0]+deletion:
                D[i+1][j+1][1] = 'D'
            elif D[i+1][j+1][0] == D[i+1][j][0]+insertion:
                D[i + 1][j + 1][1] = 'I'
            else:
                if S[i] == T[j]:
                    D[i + 1][j + 1][1] = 'M'
                else:
                    D[i + 1][j + 1][1] = 'R'

    return D


def backtrace(S, T, insertion, deletion, replacement):
    D = edit_distance_without_transcript(S, T,insertion, deletion, replacement)
    # backtracing for optimal route
    m = len(S)
    n = len(T)
    i, j = m, n
    string = ''
    # we iterate until both strings have length zero, at each index we use the pointer
    # to determine which operation was taken, and backtrace accordingly
    while i > 0 or j > 0:
        string += D[i][j][1]
        # replace or matching
        if D[i][j][1] == 'R' or D[i][j][1] == 'M':
            i -= 1
            j -= 1
        # insert
        elif D[i][j][1] == 'I':
            j -= 1
        # delete
        else:
            i -= 1
    return string[::-1]


\end{lstlisting}
\newpage
\subsection*{iii) edit\textunderscore distance\textunderscore with\textunderscore gaps.py}
\lstdefinestyle{myCustomMatlabStyle}{
  language=Python,
  stepnumber=1,
  numbersep=10pt,
  tabsize=4,
  showspaces=false,
  showstringspaces=false
}
\lstset{basicstyle=\footnotesize,style=myCustomMatlabStyle}
\begin{lstlisting}
# S, T are the two strings to be compared
# weight=w(l)=u
def edit_distance_without_transcript(S, T, weight, scoring):
    m = len(S)  # length of S
    n = len(T)  # length of T
    # initialising the results table
    Vgap = [[[0, ''] for i in range(n+1)] for i in range(m+1)]
    E = [[0 for i in range(n+1)] for i in range(m+1)]
    F = [[0 for i in range(n+1)] for i in range(m+1)]
    # boundary conditions
    for i in range(m+1):
        Vgap[i][0][0] = weight
        Vgap[i][0][1] = 'D'
        E[i][0] = weight    # so E(i,1)=Vgap(i,0)+u
    for j in range(n+1):
        Vgap[0][j][0] = weight
        Vgap[0][j][1] = 'I'
        F[0][j] = weight   # so F(1,j)=Vgap(0,j)+u
    Vgap[0][0][0] = 0
    Vgap[0][0][1] = 'M'
    # recursion calculation
    for i in range(1, m+1):
        for j in range(1, n+1):
            # recursive formula to calculate E[i][j] and F[i][j]
            E[i][j] = max(E[i][j-1], Vgap[i][j-1][0]+weight)
            F[i][j] = max(F[i-1][j], Vgap[i-1][j][0]+weight)
            # calculate G[i][j]
            G = Vgap[i-1][j-1][0] + scoring[(S[i-1], T[j-1])]
            Vgap[i][j][0] = max(E[i][j], F[i][j], G)
            if Vgap[i][j][0] == G:      # G is replacement or matching
                if S[i-1] == T[j-1]:
                    Vgap[i][j][1] = 'M'
                else:
                    Vgap[i][j][1] = 'R'
            elif Vgap[i][j][0] == E[i][j]:
                Vgap[i][j][1] = 'I'     # E is insertion
            else:
                Vgap[i][j][1] = 'D'     # F is deletion
    return Vgap


def backtrace(S, T, weight, scoring):
    Vgap = edit_distance_without_transcript(S, T, weight, scoring)
    # backtracing for optimal route
    m = len(S)
    n = len(T)
    i, j = m, n
    string = ''
    # we iterate until both strings have length zero, at each index we use the pointer
    # to determine which operation was taken, and backtrace accordingly
    while i >= 0 or j >= 0:
        string += Vgap[i][j][1]
        # replace or matching
        if Vgap[i][j][1] == 'R' or Vgap[i][j][1] == 'M':
            i -= 1
            j -= 1
        # insert
        elif Vgap[i][j][1] == 'I':
            j -= 1
        # delete
        else:
            i -= 1
    return string[::-1]
\end{lstlisting}
\newpage
\subsection*{iv) limit \textunderscore score.py}
\lstdefinestyle{myCustomMatlabStyle}{
  language=Python,
  stepnumber=1,
  numbersep=10pt,
  tabsize=4,
  showspaces=false,
  showstringspaces=false
}
\lstset{basicstyle=\footnotesize,style=myCustomMatlabStyle}
\begin{lstlisting}
import random
from edit_distance_with_gaps import *

m = 1000
p = 0.5
u = -3
n = 1000
U_n = ['' for i in range(m)]
V_n = ['' for i in range(m)]
Vgap = [0 for i in range(m)]
s = {('a', 'b'): -1, ('a', 'a'): 1, ('b', 'a'): -1, ('b', 'b'): 1}

for i in range(m):
    for j in range(n):
        if random.random() < p:
            U_n[i] += 'a'
        else:
            U_n[i] += 'b'
        if random.random() < p:
            V_n[i] += 'a'
        else:
            V_n[i] += 'b'
    Vgap[i] = edit_distance_without_transcript(U_n[i], V_n[i], u, s)[-1][-1][0]
estimated_mean = sum(Vgap)/m
result = estimated_mean/n

print(result)

\end{lstlisting}
\bigskip\bigskip
\subsection*{v) edit\textunderscore distance\textunderscore with\textunderscore local\textunderscore alignment.py}
\lstdefinestyle{myCustomMatlabStyle}{
  language=Python,
  stepnumber=1,
  numbersep=10pt,
  tabsize=4,
  showspaces=false,
  showstringspaces=false
}
\lstset{basicstyle=\footnotesize,style=myCustomMatlabStyle}
\begin{lstlisting}
# S, T are the two strings to be compared
# deletion is s(Si,-)
# insertion is s(-,Tj)
def v_sub(S, T, insertion, deletion, replacement):
    m = len(S)  # length of S
    n = len(T)  # length of T
    # initialising the results table
    Vsfx = [[0 for i in range(n+1)] for i in range(m+1)]
    maximum_score = 0
    # boundary conditions
    for i in range(m+1):
        Vsfx[i][0] = 0
    for j in range(n+1):
        Vsfx[0][j] = 0
    # recursion calculation
    for i in range(1, m+1):
        for j in range(1, n+1):
            Vsfx[i][j] = max(0, Vsfx[i][j-1]+deletion, Vsfx[i-1][j]+insertion,
                             Vsfx[i-1][j-1]+replacement[(S[i-1], T[j-1])])
            # update the maximum score if required
            if Vsfx[i][j] > maximum_score:
                maximum_score = Vsfx[i][j]
    return maximum_score

\end{lstlisting}
\end{document}